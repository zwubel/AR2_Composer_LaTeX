\section{Ausblick}\label{sec:ausblick}\todo[inline, color=green]{Lukas}
Während dem Projektverlauf von \emph{MArC} konnten einige Ideen für die Umsetzung der Anwendung reifen, welche aus Zeitgründen im vorliegenden Projekt nicht realisiert werden konnten. Dabei handelt es sich sowohl um Ideen, welche ursprünglich in der Projektplanung als Bestandteile von \emph{MArC} vorgesehen waren, als auch um solche, die während der Arbeit am Projekt entstanden. Die folgenden Abschnitte beschäftigen sich mit den vielversprechendsten dieser Ideen. Diese können besonders dann, wenn nach dem Projektabschluss in einem anderen Rahmen noch weiter am Projekt gearbeitet werden sollte, dem nachfolgenden Team als Anhaltspunkte dienen.
\subsection{Trackingmethoden}\todo[inline, color=green]{Vera}
\todo[inline, color=red]{Laura}
In der Zukunft kann die Tracking Software noch weiter verbessert werden. Zum einen kann geprüft werden, ob es möglich ist die OBBs der grünen Rechtecke auch mit einer Bewegungsabschätzung, wie zum Beispiel dem Kalman Filter \cite{article:Kalman} oder dem $\alpha$/$\beta$/$\gamma$ Filter \cite{article:alphabetagamma}, zu verfolgen. Doch an dieser Stelle gilt es auch zu prüfen, ob dann die Performanz der Tracking-Anwendung noch ausreichend für das Tracking mit der gewünschten Bildrate ist. Zur Verbesserung der gesamten Performanz kann auch die Möglichkeit geprüft werden die Implementierung der Verfolgung und Detektion zu parallelisieren. Gelingt dies, so kann unter Umständen auch eine Kamera mit höherer Bildrate eingesetzt werden.\\
Um auch bei schnellen Bewegungen einen abhängigen Winkel $\omega_{WM}$ (vgl. Abschnitt \ref{sec:MarkerObjekte}) zu ermitteln, kann auf dem Würfel Marker der untere linke Eckpunkt des grünen Rechtecks in einer anderen Farbe eingefärbt werden. Dies erfordert ein zusätzliches Keying um diese Ecke zu identifizieren und mit den generierten Eckpunkten abzugleichen.

\subsection{Erweiterung von \emph{MArC} als Augmented-Reality-System} \label{sec:PlanedWebcam}
%\todo[inline, color=green]{geschrieben: Laura}
%\todo[inline, color=red]{korrigiert: Lukas}
Wie aus dem Projektmanagement in Kapitel \ref{sec:pm} hervorgeht, war zunächst geplant, dass \textit{MArC} als AR-Anwendung umgesetzt wird. So wie das System implementiert ist, ist die Möglichkeit nicht gänzlich verworfen worden. So ist es durchaus denkbar, nach dem Abschluss des vorliegenden Projekts die Systembasis von Virtual-Reality auf Augmented-Reality zu ändern. Hierzu müsste zunächst eine geeignete Stereo-Kamera an das \textit{HTC Vive} Head-Mounted Display angebracht werden. Dabei ist darauf zu achten, dass man einen Kompromiss findet, sodass die Kamera zwar möglichst auf Augenhöhe des Benutzers am HMD befestigt wird, aber trotzdem möglichst wenige Sensoren des \emph{HTC}-Trackingsystems verdeckt werden. Zudem muss bedacht werden, dass der \textit{Leap Motion}-Controller ebenfalls Platz unterhalb der Kamera in Anspruch nimmt. Der Kalibrierungsansatz aus Kapitel~\ref{sec:calib} müsste ebenfalls entsprechend angepasst und gegebenenfalls erweitert werden. 

\subsection{Export von Architekturdaten}
%\todo[inline,color=green]{Paul}
%\todo[inline, color=red]{Lukas}
Eine weitere Erweiterungsmöglichkeit von \emph{MArC} ist ein alternativer Export der Architekturdaten, wie sie für jedes Gebäude in dessen Kontextmenü angezeigt werden.\\
Hat der Nutzer zum Beispiel eine Siedlung errichtet und diese gespeichert, wäre es denkbar, dass der Benutzer die Gebäudedaten in tabellarischer Form benötigt. Hierfür wäre es möglich die Gebäudedaten, die innerhalb der Szene auf den Kontextmenüs dargestellt werden zu exportieren. Denkbar wäre etwa eine übersichtliche \emph{Excel}-Tabelle, innerhalb derer die Gebäude eindeutig benannt sind und dann deren Informationen kompakt zusammengefasst darunter notiert sind. Des weiteren könnten bestimmte Daten der einzelnen Gebäude bereits in einer Art Zusammenfassung dargestellt werden. Denkbar wäre auch, eine Excel Tabelle mit mehreren Arbeitsmappen zu erstellen, wenn mehrere Entwürfe für eine Siedlung zur Diskussion gestellt werden. Dann würde für jeden Entwurf eine Arbeitsmappe zur Verfügung stehen, innerhalb derer die Daten notiert sind. Ein direkter Datenvergleich der verschiedenen Siedlungsbauten wäre damit problemlos möglich.

\subsection{Evaluierung von \emph{MArC}}\todo[inline, color=green]{Lukas}
\todo[inline, color=red]{Paul}
Zur Erreichung des Gesamtziels einer jeden Produktentwicklung~--~nämlich ein markt\-rei\-fes und kommerziell verwertbares Produkt zu erhalten~--~gehört für gewöhnlich auch die Beurteilung der Entwicklung mittels Benutzerstudien. Nur so lässt sich häufig eine Aussage darüber treffen, ob das Produkt in den Händen von unvoreingenommenen Testpersonen den gewünschten Zweck weiterhin erfüllt.\\
Eine solche Evaluierung wird für \emph{MArC} als einen der weiteren Schritte nach Beendigung des vorliegenden Projekts ebenfalls empfohlen. Vor der Durchführung von Nutzerstudien sind sicher noch weitere Schritte nötig, um das "`Produkt"' \emph{MArC} als ebensolches abzurunden, aber schlussendlich ist eine Evaluierung durch potenzielle Endkunden ununmgänglich. Im Falle von \emph{MArC} ist der Nutzen für die Zielgruppe der Architekten nämlich keineswegs selbstverständlich. Neben dem theoretischen Nutzen, den ein perfekt funktionierendes System mit den gegebenen Eigenschaften bringen würde, sollte vor allem der tatsächliche Nutzen Gegenstand einer Untersuchungsreihe sein. Denn angenommen der theoretische Nutzen des Produktes ist immens, so könnte die unvorteilhafte Realisierung des Systems dennoch den tatsächlichen Nutzen erheblich einschränken. Dies ist insbesondere vor dem Hintergrund der Interaktionsmethoden in Mixed-Reality-Systemen zu sehen, da diese~--~je nach Anwendung~--~keineswegs grundsätzlich intuitiv und effizient sind. \textcolor{red}{Vll kann man hier noch kurz darauf eingehen, welche Art der Evaluierung sinnvoll sind } 