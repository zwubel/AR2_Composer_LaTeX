\section{Ausblick}\label{sec:ausblick}\todo[inline]{Lukas}
\subsection{Trackingmethoden}\todo[inline, color=green]{Vera}
\todo[inline, color=blue]{Laura}
In der Zukunft kann die Tracking Software noch weiter verbessert werden. Zum einen kann geprüft werden, ob es möglich ist die OBB der grünen Rechtecke auch mit einer Bewegungsabschätzung wie zum Beispiel dem Kalman Filter \cite{article:Kalman} oder dem $\alpha$/$\beta$/$\gamma$ Filter \cite{article:alphabetagamma} zu verfolgen. Doch an dieser Stelle gilt es auch zu prüfen ob die Performanz der Tracking Anwendung noch ausreichend für die Verfolgung mit der gewünschten Bildrate ist. Zur Verbesserung der gesamten Performanz kann auch die Möglichkeit geprüft werden die Implementierung der Verfolgung und Detektion zu parallelisieren. Gelingt dies kann unter Umständen auch eine Kamera mit höherer Bildrate eingesetzt werden.
Um auch bei schnellen Bewegungen einen abhängigen Winkel $\omega_{WM}$ zu ermitteln, kann auf dem Würfel Marker der untere linke Eckpunkt des grünen Rechtecks in einer anderen Farbe eingefärbt werden. Dies erfordert ein zusätzliches Keying um diese Ecke zu identifizieren und mit den generierten Eckpunkten abzugleichen.

\subsection{AR Erweiterung mit der Webcam} \label{sec:PlanedWebcam}\todo[inline, color=green]{Laura}
\todo[inline, color=blue]{Lukas}
Wie aus dem Projektmanagement in Kapitel \ref{sec:pm} hervorgeht, war zunächst geplant, dass \textit{MArC} als AR-Anwendung umgesetzt wird. So wie das System implementiert ist, ist die Möglichkeit nicht gänzlich verworfen worden und es ist durchaus realistisch in Zukunft von VR auf AR umzusteigen. Hierzu müsste man zunächst eine geeignete Stereo-Kamera an die \textit{HTC Vive} anbringen. Dabei ist darauf zu achten, dass man einen Kompromiss findet, sodass die Kamera zwar möglichst auf Augenhöhe des Benutzers an dem HMD befestigt wird, aber trotzdem möglichst wenige Sensoren verdeckt. Zudem muss bedacht werden, dass der \textit{Leap Motion}-Controller noch unterhalb der Kamera angebracht werden können muss. Der Kalibrierungsansatz aus Kapitel~\ref{sec:calib} müsste ebenfalls entsprechend angepasst bzw. erweitert werden. 

\subsection{Daten als Excel Tabelle}\todo[inline,color=green]{Paul}
\todo[inline, color=blue]{Lukas}
Eine weitere Erweiterungsmöglichkeit ist ein alternativer Datenexport. Hat man Beispielsweise eine Siedlung errichtet und diese gespeichert, könnte der Benutzer die Gebäudedaten in tabellarischer Form gebrauchen. Hierfür wäre es möglich die Gebäudedaten, die innerhalb der Szene auf den Contextmenus dargestellt weden zu exportieren. Denkbar wäre z.B. eine übersichtliche Excel Tabelle, innerhalb derer die Gebäude eindeutig benannt sind und dann deren Informationen kompakt zusammengefasst darunter notiert sind. Denkbar wäre auch, eine Excel Tabelle mit mehreren Arbeitsmappen zu erstellen, wenn mehrere Entwürfe für eine Siedlung zur Diskussion gestellt werden. Dann wäre für jeden Entwurf eine Arbeitsmappe verfasst, innerhalb derer die Daten notiert sind. Ein direkter Datenvergleich der verschiedenen Siedlungsbauten wäre dann problemlos möglich.

\subsection{Evaluierung von \emph{MArC}}\todo[inline, color=green]{Lukas}
\todo[inline, color=red]{Paul}
Zur Erreichung des Gesamtziels einer jeden Produktentwicklung~--~nämlich ein markt\-rei\-fes und kommerziell verwertbares Produkt zu erhalten~--~gehört für gewöhnlich auch die Beurteilung der Entwicklung mittels Benutzerstudien. Nur so lässt sich häufig eine Aussage darüber treffen, ob das Produkt in den Händen von unvoreingenommenen Testpersonen den gewünschten Zweck weiterhin erfüllt.\\
Eine solche Evaluierung wird für \emph{MArC} als einen der weiteren Schritte nach Beendigung des vorliegenden Projekts ebenfalls empfohlen. Vor der Durchführung von Nutzerstudien sind sicher noch weitere Schritte nötig, um das "`Produkt"' \emph{MArC} als ebensolches abzurunden, aber schlussendlich ist eine Evaluierung durch potenzielle Endkunden ununmgänglich. Im Falle von \emph{MArC} ist der Nutzen für die Zielgruppe der Architekten nämlich keineswegs selbstverständlich. Neben dem theoretischen Nutzen, den ein perfekt funktionierendes System mit den gegebenen Eigenschaften bringen würde, sollte vor allem der tatsächliche Nutzen Gegenstand einer Untersuchungsreihe sein. Denn angenommen der theoretische Nutzen des Produktes ist immens, so könnte die unvorteilhafte Realisierung des Systems dennoch den tatsächlichen Nutzen erheblich einschränken. Dies ist insbesondere vor dem Hintergrund der Interaktionsmethoden in Mixed-Reality-Systemen zu sehen, da diese~--~je nach Anwendung~--~keineswegs grundsätzlich intuitiv und effizient sind. \textcolor{red}{Vll kann man hier noch kurz darauf eingehen, welche Art der Evaluierung sinnvoll sind } 