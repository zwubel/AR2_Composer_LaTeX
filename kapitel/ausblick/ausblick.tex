\section{Ausblick}
\subsection{Trackingmethoden}\todo[inline, color=green]{Vera}
In der Zukunft kann die Tracking Software noch weiter verbessert werden. Zum einen kann geprüft werden, ob es möglich ist die OBB der grünen Rechtecke auch mit einer Bewegungsabschätzung wie zum Beispiel dem Kalman Filter \cite{article:Kalman} oder dem $\alpha$/$\beta$/$\gamma$ Filter \cite{article:alphabetagamma} zu verfolgen. Doch an dieser Stelle gilt es auch zu prüfen ob die Performanz der Tracking Anwendung noch ausreichend für die Verfolgung mit der gewünschten Bildrate ist. Zur Verbesserung der gesamten Performanz kann auch die Möglichkeit geprüft werden die Implementierung der Verfolgung und Detektion zu parallelisieren. Gelingt dies kann unter Umständen auch eine Kamera mit höherer Bildrate eingesetzt werden.
Um auch bei schnellen Bewegungen einen abhängigen Winkel $\omega_{WM}$ zu ermitteln, kann auf dem Würfel Marker der untere linke Eckpunkt des grünen Rechtecks in einer anderen Farbe eingefärbt werden. Dies erfordert ein zusätzliches Keying um diese Ecke zu identifizieren und mit den generierten Eckpunkten abzugleichen.
\subsection{AR Erweiterung mit der Webcam}\todo[inline]{Laura}
\subsection{Daten als Excel Tabelle}\todo[inline,color=green]{Paul}
Eine weitere Erweiterungsmöglichkeit ist ein alternativer Datenexport. Hat man Beispielsweise eine Siedlung errichtet und diese gespeichert, könnte der Benutzer die Gebäudedaten in tabellarischer Form gebrauchen. Hierfür wäre es möglich die Gebäudedaten, die innerhalb der Szene auf den Contextmenus dargestellt weden zu exportieren. Denkbar wäre z.B. eine übersichtliche Excel Tabelle, innerhalb derer die Gebäude eindeutig benannt sind und dann deren Informationen kompakt zusammengefasst darunter notiert sind. Denkbar wäre auch, eine Excel Tabelle mit mehreren Arbeitsmappen zu erstellen, wenn mehrere Entwürfe für eine Siedlung zur Diskussion gestellt werden. Dann wäre für jeden Entwurf eine Arbeitsmappe verfasst, innerhalb derer die Daten notiert sind. Ein direkter Datenvergleich der verschiedenen Siedlungsbauten wäre dann problemlos möglich.

\subsection{Validierung}\todo[inline]{Luke}