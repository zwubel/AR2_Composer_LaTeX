\section{Projektmanagement} \label{sec:pm}

Dieses Kapitel beschreibt die Planung und das Management des Projektes \textit{MArC}. Es ist in die einzelnen Projektphasen aufgegliedert, diese enthalten wiederum die wichtigen Punkte der entsprechenden Phasen.

\subsection{Projektdefinition}
\todo[inline,color = green ]{Paul}
\subsubsection{Problemanalyse}
\label{sec:problemanalye}
Bisher erfolgte die Visualisierung von Architekturprojekten vorwiegend mit (z.B. aus Pappkarton oder Stypropor) aufwendig hergestellten Modellen. Diese werden im Laufe des Prozesses immer wieder verändert und verbessert.
Um eine Darstellung von Modellen innerhalb einer bestehenden Umgebung zu ermöglichen, werden abstrahierte, ungenaue Bodenmodelle aus einzelnen Pappschichten geschnitten und übereinander geklebt. Jede Änderung des Entwurfs zieht viel Arbeit nach sich. Ein einfaches Arrangieren und ein Testen auf Licht und Schattenwurf ist nahezu nicht möglich.\\
Das \textit{MArC} Projektteam stellte den Kontakt zu einem Architekturbüro in Köln her um deren Planungsvorgehen zu sehen und nachvollziehen zu können. Anhand dieser Analyse stellten sich die Projektziele heraus.


\subsubsection{Projektziele und Anforderungen}
Ziel des Projektes \textit{MArC} ist es, Architekten eine neuartige Darstellung von Gebäuden und Objekten zu erlauben. Auf einem mit Kameras versehenen Tisch werden Marker in abstrahierter Form der darzustellenden Objekte platziert. Mittels Augmented oder Virtual Reality werden mit Hilfe einer VR-Bille (Oculus Rift) die Gebäude und Objekte aus den CAD Dateien an den Stellen gerendert, an denen die Marker auf dem Tisch gestellt wurden.
Es ist möglich, diese dann über die mittels Kamera getrackten Marker händisch zu manipulieren und somit eine Szene zu errichten, die aus allen Blickwinkeln betrachtet werden kann und jeder Zeit durch weitere Objekte ergänzt werden kann. \\
Das Projekt soll im vier Personen Team durchgeführt werden. Hierfür stehen 12 ECTS Punkte pro Person zur Verfügung, also 48 ECTS bei vier Personen. Die Arbeit soll sich also innerhalb von 1440 Arbeitsstunden insgesamt umsetzen lassen.


\subsubsection{Lösungskonzept}
\textit{MArC} soll mit Hilfe von Augmented oder Virtual Reality eine intuitive und flexible Arbeit mit einfachen Modellen ermöglichen. So kann per Augmented Reality Brille ein Modell auf einem Tisch visualisiert und grundlegend manipuliert werden. 
Anordnungen von mehreren Modellen in einer Umgebung (z.B. Siedlung) können exportiert und so direkt weiterverarbeitet werden.
Durch die VR-Brille mit montierten stereoskopischen Kameras wird ein Abbild der eralen Umgebung gezeigt. Auf einem Tisch liegende Würfel werden durch das innovative System getrackt; diese können von dem Benutzer intuitiv verschoben und arrangiert werden. \\
Das Terminziel ist das Ende des Wintersemesters 2016/2017. Die Projektabgabe soll am 28.2.2017 fertig gestellt worden sein. Die Dokumentation ist am 31.3.2017 Fertigzustellen und die Präsentation am 10.5.2017, wobei sich dieser Termin im laufe der Arbeit ergab.

 
\subsubsection{Durchführbarkeitsanalyse}
In diesem Abschnitt soll die Durchführbarkeit des Projektes erklärt werden. Diese Durchführbarkeitsanalyse stützt sich auf die folgenden Analysen:
\begin{itemize}
	\item Technologische Durchführbarkeit
	\item  Ökonomische Durchführbarkeit
	\item Rechtliche Durchführbarkeit
	\item  Operationale Durchführbarkeit
	\item  Terminliche Durchführbarkeit
\end{itemize}

\paragraph{Technologische Durchführbarkeit} Für die Technologische Durchführbarkeit stehen sowohl die Technischen Möglichkeiten, als auch das know how der Mitarbeiter im Zentrum. \textit{MArC} nutzt neuartige Technologien, die einerseits zum Teil in der Hochschule vorhanden waren, teilweise jedoch aus Übersee nachgekauft werden mussten. So war die HTC Vive, das HMD das Mittel zur Wahl was die Darstellung der virtuellen Elemente angeht. Diese war bereits in der Hochschule vorhanden und es gab eininge Komilitonen, die bereits damit gearbeitet hatten. Das Projektteam konnte also sporadisch auf bisherige Erfahrungen zurückgreifen. Ebenso verhielt es sich mit der Leap Motion, die bereits in der TH Verfügbar war und ebenfalls schon in Projekten verwendet wurde. Auch hier konnte auf das Wissen sporadisch zurückgegriffen werden um eine bessere Machbarkeitsabschätzung zu erlangen. In beiden Fällen stellte sich die Idee diese Geräte zu verwenden als gute Lösung heraus, unter anderem wegen bisher geglückter Projekte.\\
Anders verhielt es sich jedoch mit der OVR Vision. Dieses Gerät war nicht in der Hochschule verwendet und keines der Teammitglieder hatte Erfahrung mit dem Gerät. Da jedoch die anderen Geräte funktional mit guter Erfahrung zur Verfügung standen, entschied sich das Team die OVR Vision zu kaufen und das Projekt darauf aufzubauen.\\
Ziel der Verwendung der OVR Vision war es, aus der Virtuellen Realität mit Haptischer Komponente eine Augmented Reality Anwendung zu schaffen, die dem Benutzer die Darstellung der virtuellen Elemente in seiner Umgebung darstellte.\\
Für das Tracking sollte eine bereits vorhandene Industriekamera von UEye verwendet werden. Diese wurde ebefalls in vielen Bereichen und verschiedenen Fächern in der Hochschule genutzt und war bereits bekannt.\\
Desweiteren konnte das Team auf weitreichendes individuelles Wissen zurückgreifen. So waren unter anderen bereits eine Bachelorarbeit mit \textit{Unity} und einem HMD erarbeitet worden, andere Teammitgleider hatten hingegen tiefgreifendes Wissen über Mustererkennung und digitale Bildverarbeitung, die für das Markertracking unumgänglich sind. Fehlend zu diesem Wissen war jedoch das Anwenden von virtuellen Szenerien in Augmentierten Situationen. Dies ist sowohl technisch Neuland, im Laufe des Projektes startete Microsoft den Verkauf der \textit{Hololens}, einem echten Augmented Reality Headmounted Display, als auch für die Projektmitglieder. So wurden im Fach "Virtuelle und Erweiterte Realität", welches alle der Mitglieder schon bereits vor dem Projektbeginn belegten, wichtiges Knowhow für Augmented Reality Situationen erst im letzten Semester zu der Thematik ausführlich hinzugefügt.

\paragraph{Ökonomische Durchführbarkeit} Neben der Technologischen Durchführbarkeit war die Ökonomische Durchführbarkeit zu erörtern. Durch die Rahmenbedingung, dass das Projekt innerhalb des Studiums umgesetzt wird und keine Gehälter zu zahlen sind, verringern sich die Kosten auf die zuzulegende Hardware und benötigter Zubehör. Diese Summen sind innerhalb des bereitgestellten Budgeds bezahlbar. Wichtiger jedoch sind die Zeitkosten für die einzelnen Mitglieder. So war hat jedes Mitglied ein eigenverantwortliches Zeitmanagement durchzuführen und genügend Zeit für das Projekt zur Verfügung zu stellen. Da das Projekt nicht als Vollzeitprojekt geplant ist, hat jeder Mitarbeiter seine anderen Studienfächer nebenher. Da der Zeitraum über zwei Semester jedoch viel Zeit bietet, war das Projekt im vorgegebenen Ökonomischen Rahmen machbar.

\paragraph{Rechtliche Durchführbarkeit} Dieser Punkt sei zur Vollständigkeit genannt. Da es sich bei \textit{MArC} um ein studentisches Projekt handelt welches nicht verkauft werden sollte, ist das Projekt in seinem geplanten Umfang rechtlich durchführbar.

\paragraph{Operationale Durchführbarkeit} Da Jedes Teammitglied individuelles, tiefgreifendes Wissen mit in das Projekt bringt und alle Mitglieder bereits im Bereich Virtual Reality gearbeitet haben, ist großes Potential für das Projekt vorhanden. Lediglich die Weiterführung von Virtueller in Agmentierter Reality ist für die Mitglieder neu und muss zusätzlich erlernt werden. Durch den Zeitrahmen ist die Operationale Durchführbarkeit gut.


\paragraph{Terminliche Durchführbarkeit} Insgesamt wurden für das Projekt 12ECTS x 4Personen x 30Stunden = 1440 Arbeitsstunden eingerechnet. Innerhalb diesen Rahmens ist das Projekt bis zum Ende des Wintersemesters 2016/2017 gemeinsam fertig zu stellen. Terminlich ist bei aufgeteilter Arbeitslast ein Projekt wie \textit{MArC} machbar, wobei diese Machbarkeit mit dem betreuuenden Professor abgesprochen und auf dessen Erfahrungen zurückgegriffen wurde.

\subsubsection{Projektauftragsformular}
Das Projektauftragsformular ist in \ref{tab:Projektformular} zu finden.
\begin{table}
	\centering
	\begin{tabularx}{\textwidth}{|l|l|}
	\hline
	\Absatzbox{}
	& \textbf{Projektauftrag}\\	
	\hline
	Projektname & MArC ( Mixed Reality Architecture Composer)\\
	\hline
	Projektleiter & Paul Berning, Lukas Kolhagen\\
	\hline
	Projektanlass & Masterprojekt Master Medientechnologie\\
	\hline
	Projektziele & Erstellung einer Augmented bzw. Virtual Reality \\
	&Anwendung, mithilfe derer Architekten Siedlungen\\
	& bauen und einfach nachträglich bearbeiten können.\\
	 &Dabei soll der Nutzer Gebäude mittels getrackter\\
	& Würfel verschieben und rotieren können \\
	&und sämtliche Interaktion (manipulation der Größe\\ 
	& der Gebäude, Menüiteraktionen) mit den Fingern \\
	&durchführen können.\\
	\hline
	Zu erarbeitende Ergebnisse & Abgeliefert werden zwei \\
	&lauffähige .exe Datei, die auf der einen Seite das\\
	&Tracking auf einem Computer, und auf dem \\
	&anderen die Darstellung ermöglichen. \\
	&Eigentümer der Hardware ist zu jeder Zeit\\
	&der Auftraggeber. Das Projektteam ist dafür\\
	&verantwortlich, dass die Software im vorgegebenen\\
	&Rahmen funktioniert. Es wird dazu eine Readme \\
	&Datei ausgeliefert, die das Verwenden der\\
	& Programme erklärt. Eine ausführliche \\
	&Dokumentation ist bis zum 30.3.2017 abzuliefern,\\
	& eine Abschlusspräsentation mit allen Ergebnissen\\
	& ist am 10.5.2017 zu halten. Zusätzlich erstellt\\
	 &das Projektteam ein Video welches das\\
	& Lauffähige Ergebnis dokumentiert.\\
	
\hline
	Projektbudget & Das Projekt unterliegt einem\\
	& monetären Budget von 500 Euro und einem\\
	& zeitlichen Budged von 4Personen x \\
	&12ECTS x 30 Stunden = 1440 Stunden.\\
	\hline
	Randbedingungen & Sämtliche Öffnungszeiten\\
	& der Räumlichkeiten der Arbeitgeber, Verfügbarkeit\\
	& des begleitenden Professors für Termine nur nach \\
	&Absprache, einhalten von Projektbudget und \\
	&Zeitrahmen, insbesondere des individuellen \\
	&Zeitmanagements der Mitglieder wegen anderer\\
	& Universitärer Veranstaltungen\\
	\hline
	Termine und Meilensteine & Projektabgabe: 28.2.2017, \\
	&Abgabe der Dokumentation: 30.3.2017, \\
	&Projektpräsentation und Abgabe des Videos: \\
	&10.5.2017\\
	\hline

	\end{tabularx}
	\caption{Identifikation der Arbeitspakete und Meilensteine}
	\label{tab:Projektformular}
\end{table}

\subsubsection{Projektorganisation}
\paragraph{Leitung} Die Leitung des Teams wurde aufgeteilt zwischen Lukas Kolhagen und Paul Berning. Die Ursprünglich gedachte Aufteilung in die beiden Semester ist einer durchgängigen Kooperation und Absprache gewichen, welche das Team organisatorisch stützte.
\paragraph{Team} Das Team teilte sich des Weitern in zwei Unterteams auf. Basierend auf den unterschiedlichen individuellen Kompetenzen wurde das Tracking der Marker hauptsächlich von Laura Anger und Vera Brockmeyer übernommen. Auf der anderen Seite konnten Lukas Kolhagen und Paul Berning fundierte Unity Kenntnisse aufweisen und kümmerten sich hauptsächlich um die visuelle Darstellung der Applikation.

\paragraph{Infrastruktur} Als Arbeitsplatz wurde Seitens der Hochschule ein Raum zur Verfügung gestellt, der über eine fest installierte HTC Vive und einem dazugehörigen performanten Computer verfügt. In diesem wurde das Projekt durchgeführt. Optional standen dem Team weitere Performante Laptops zur Verfügung und notwendige zusätzliche Hardware und Zubehör wurde im Laufe der Zeit hinzu gekauft oder waren schon Seitens der Hochschule Vorhanden und verfügbar gemacht.
- Informationssystem

\subsection{Projektplanung}
\todo[inline,color = green ]{Paul}

\subsubsection{Arbeitspakete}
Wie in größeren Projekten üblich, wurde die Arbeit in mehrere Arbeitspakete (AP) und Meilensteine (MS) segmentiert. Die detaillierte Auflistung dieser kann in Tabelle \ref{tab:APMS} gefunden werden.
\begin{table}
	\centering
	\begin{tabular}{|l|l|l|l|}
		\hline
		\Absatzbox{}
		\textbf{Paketname (AP) / Meilenstein (MS)}& \textbf{Start} &  \textbf{Ende} &  \textbf{Verantwortl.:} \\
		\hline
		AP1 – Organisation und Planung & 21.03.2016 & 10.05.2017 & PB/LK\\		
  		\hline
		AP2 - Trackingalgorithmus & 2.5.2016 & 5.2.2017 & VB/LA\\
		\hline
		MS2.01 Recherche  & 2.5.2016 & 31.5.2016 & VB/LA\\
 	 	\hline
		MS2.02 Implementierung  & 1.6.2016 & 1.7.2016 & VB/LA\\
		\hline
		AP3 - Framework  & 24.5.2016 & 13.7.2016 & PB/LK\\
		\hline
		MS 3.01 Unity und Github einrichten  & 24.5.2016 & 31.5.2016 & PB/LK\\
		\hline
		MS 3.02 HTC Vive anbinden & 13.7.2016 & 16.11.2016 & PB/LK\\
		\hline
		MS 3.03 Trackingkamera besorgen &16.06.2016 & 1.7.2016 & PB/LK\\
		\hline
		MS  3.04 Stereokamera einrichten &1.7.2016 & 13.7.2016 & PB/LK\\
		\hline
		AP4 - Hardware & 31.5.2016 & 13.7.2016 & PB/LK/LA\\
		\hline
		MS 4.01 Würfelmarker erstellen & 2.5.2016 & 31.5.2016 & LA\\
		\hline
		MS 4.02 Testbilder für Tracking erstellen  & 30.5.2016 & 17.6.2016 & PB/LK\\
		\hline
		MS 4.03 Tracking Kamera Montage	& 1.6.2016 & 15.6.2016 & PB/LK\\
		\hline
		MS 4.04 PC besorgen und einrichten & 30.5.2016 & 17.6.2016 & PB/LK\\
		\hline
		AP5 - CAD Import/Export & 1.8.2016 & 13.9.2016 & PB/LK\\
		\hline
		MS 5.01 CAD/ 3D Formate festlegen & 1.8.2016 & 9.8.2016 & PB/LK\\
		\hline
		MS 5.02 3D Import implementieren & 9.8.2016 & 13.9.2016 & PB/LK\\
		\hline
		AP6 - Testing & 10.7.2016 & 27.11.2016 & TEAM\\
 		\hline
		MS 6.01 Usability intern & 10.7.2016 & 15.7.2016 & TEAM\\
 		\hline
		MS 6.02 Usability mit Experten &22.11.2016 & 27.11.2016 &TEAM\\
 		\hline
		AP7 - Dokumentation &1.6.2016 & 31.3.2017 & TEAM\\
 		\hline
		MS 7.01 Schriftliche Ausarbeitung & 1.6.2016 & 31.3.2017 & TEAM\\
 		\hline
		MS 7.02 Abschlusspräsentation & 24.4.2017 & 10.5.2017 & TEAM\\
 		\hline
		AP8 - Prototypen & 2.5.2016 & TERMIN & TEAM\\
 		\hline
		MS 8.1 Prototyp1 (Paper Mockup) & 2.5.2016 & 2.5.2016 & TEAM\\
 		\hline
		MS 8.2 Prototyp2 (runnable) & 3.5.2016 & 13.7.2016 & TEAM\\
 		\hline
		MS 8.3 Prototyp3 (last prototype) & 13.7.2016 & 16.11.2016 & TEAM\\
 		\hline
				
	\end{tabular}
	\caption{Identifikation der Arbeitspakete und Meilensteine}
	\label{tab:APMS}
\end{table}

\subsubsection{Projektstrukturplan}
Der Projektstrukturplan ist dem Anhang beigefügt.


\subsubsection{Ablaufplan, Terminplan (Gantt Chart)}
Der Ablaufplan und Terminplan wurde der Übersichtshalber mit dem Plan über die verschiedenen Arbeitspakete und Meilensteine in Tabelle \ref{tab:APMS} aufgeführt. Das Gantt Chart ist der Übersicht halber im Anhang beigefügt
\subsubsection{Kapazitätsplan}
Der Kapazitätsplan enthält die Ressourcen über die Projektzeit von März 2016 bis März 2017. Der Plan ist tabellarisch sowie als Diagramm im Anhang zu finden.

\subsubsection{Kostenplan}
Der Kostenplan ist ebenfalls dem Anhang beigefügt.

\subsubsection{Qualitätsplan}
Der Qualitätsplan ist ebenfalls der Übersicht halber den Anhang beigefügt

\subsection{Projektdurchführung}
\todo[inline, color = green]{Paul}
\subsubsection{Kommuniktion im Team und nach Außen}
Die Kommunikation ist für den Erstellungsprozess von Projekten von hoher Wichtigkeit.\textit{ MArC} wurde innerhalb des vier Personenteams und in den kleinen zwei Personen Teams umgesetzt. Zur Kommunikation standen Email, Telefone, Whatsapp zur Verfügung. Der Datenaustausch des Projektes wurde über Github umgesetzt. Ebenfalls standen die Dienste Dropbox und Google Drive zur Verfügung.\\
Die interne Kommunikation lief zum Großteil über Whatsapp und das Telefon. Es wurden Wochenberichte erstellt, um das Team intern auf dem Stand der Dinge zu halten. Diese Mail wurde im CC jeweils auch an den Auftragsgeber gesendet.\\
Die weitere Kommunikation mit dem Auftragsgeber wurde über den Projektleiter mittels Email gehandhabt. Bei wichtigen Entscheidungen war stets das Team im CC, sodass alle Mitglieder bescheid wussten. In dringenden Fällen wurde darüberhinaus zusätzlich über Whatsapp bescheid gegeben.

\subsubsection{Maßnahmen zur Problemvermeidung}
Über die für Studienprojekte vergleichsweise lange Zeit, über die das Projekt läuft, treten zwingendermaßen Probleme auf, denen mit geeigneten Maßnahmen entgegen zu wirken ist. Die Problemlösung ist dabei eben dieser Prozess. Während der Projektlaufzeit wurde versucht auch mit kurzfristig auftretenden Problemen so umzugehen, dass das Ergebnis trotzdem so gut wie möglich ist. In diesem Abschnitt sollen einige aufgetretene Probleme exemplarisch vorgestellt werden.

\paragraph{Ausfall des Planungstools Bitrix24} Das kostenlose online Planungstool Bitrix24 wurde für die Anfängliche Planung genutzt und stand dem Team bis September 2016 zur Verfügung. Nach den Betriebsferien im September, innerhalb dieser keine Projektarbeit stattfand, also auch nicht am online Planungstool, wurde das Projekt von den Betreibern von \textit{Bitrix24} gelöscht. Ein interner Mechanismus hatte gegriffen und das Projekt als obsolet erkannt. Da alle Arbeitspakete und Meilensteine im Projektplan notiert waren, konnte das Problem gelöst werden und das Projekt ohne größere Einschränkungen weiter geführt werden. Für die Projektplanung wurde im Anschluss Excel verwendet und die Daten lokal gespeichert.

\paragraph{Ausfall von Hardware} während des Projektes fiel die \textit{OVR Vision} wegen eines technischen Defektes aus. Das führte zu dem Problem, dass kein Videobild im HMD angezeigt werden konnte, und nur die virtuelle Umgebung sichtbar war. Gleichzeitig führte der Defekt dazu, dass in einem der Arbeitslaptops die USB Ports durchgebrannt sind. Die Garantieabwicklung mit dem Hersteller gestaltete sich als schwierig, da das Gerät nach Japan verschifft werden musste. Die Gesamtdauer der Reperatur war so nicht absehbar und nicht planbar. \\
Es stand die Entscheidung an, ob auf die Reperatur gewartet werden soll und im Anschluss mit dem Gerät weiter gearbeitet werden sollte, mit dem Risiko dass es wieder kaputt geht, oder einen anderen Weg zu gehen.\\
Als Lösung spaltete sich das Team für eine Woche in zwei Kleingruppen und analysierte die Alternativen. Die eine war, die \textit{Microsoft Hololens} für das Projekt zu nutzen. Die andere, eine einfache Webcam zu verwenden.\\
Die Lösung, die \textit{Hololens} zu verwenden stellte sich als nicht praktikabel heraus. Die gesamte Applikation müsste auf die Brille geladen und dort ausgeführt werden. Eine Kommunikation über das LAN mit dem Trackingcomputer hätte über WLAN hergestellt werden müssen. Diese Lösung war zu kompliziert und zu komplex, um sie als Alternative zu verwenden.\\
Die zweite Lösung wurde jedoch weiter verwendet. Es wurde eine Webcam an das HMD angebracht und das Videobild in \textit{Unity} vor der virtuellen Kamera gerendert.\\
Diese Lösung wurde am Ende jedoch ebenfalls nicht in den letzten Prototypen eingebaut. Problem war die unzureichende Qualität der Darstellung sowie die komplizierte Kalibrierung. Da das Team mit Positionsabweichungen zwischen den echten Markern auf dem Tisch und den virtuellen zu kämpfen hatte, verwirrte ein reales Bild, welches ebenfalls nicht exakt an der soll-Position gerendert wird, nur erheblich mehr.\\
Daher entschied sich das Team am Ende, eine Mixed-Reality Anwendung als Produkt zu realisieren 


\paragraph{Markerless tracking}
 Zu Beginn des Projektes sollte die Aluminiumwürfel ohne Markierung getrackt werden (sog. Markerless tracking). Das Problem welches sich hierbei herausstellte war, dass diese Umsetzung innerhalb des Projektes zu komplex war. Ein Großteil der Projektzeit hätte auf diese Umsetzung angewendet werden müssen. Die Texturarme Marker waren nicht eindeutig identifizierbar und nach längerer Verdeckung nicht wieder zuzuordnen. Die naheliegende Lösung war, individuelle Marker zu verwenden.


\subsection{Projektabschluss}
\todo[inline, color = green]{Paul}
\subsubsection{Abschlusspräsentation}
Für das Projekt wird eine detaillierte Präsentation mit Praktischer Vorstellung des Ergebnisses erarbeitet. Diese wird am 10.5.2017 stattfinden.
 \subsubsection{Video}
Das Team wird für die Präsentation ein Video vorbereiten. Dieses stellt das Projekt vor und zeigt die Funktionalität. Es wird bis zum 10.5.2017 stattfinden. Das Team trifft sich für die Produktion zusammen in den Projekträumen und setzt diese um.


\subsection{Reflexion}\label{sec:reflexion}
\todo[inline, color=green ]{Paul }
Das Masterprojekt \textit{MArC} war mit seiner Laufzeit über zwei Semester die bislang längste praktische Projektarbeit für das gesamte Team. \\
Das Projekt wurde fertiggestellt und das Zeil erreicht, wenn auch nicht in allen Details wie vorgestellt, so ist dennoch eine funktionierende Anwendung entstanden.\\
Auf Grund mangelnder Erfahrung gab es jedoch einige Probleme, die im Nachhinein gut erkennbar und identifizierbar sind. Wichtig für die Teammitglieder ist im Nachhinein diese Fehler zu erkennen und zu bewerten, damit für etwaige spätere Projekte eben solche nicht mehr auftreten bzw. durch die Erfahrung am Anfang schon anders Bewertet werden können.\\
So stellte sich im Nachhinein heraus, dass das Team am Anfang sehr viele gute Ideen sammelte und sich viel zu hohe Ziele setzte. Mit diversen Features die dem Benutzer sehr gut gefallen würden sollte MArC von Anfang an ein Tool welches sämtliche Probleme des "analogen" Arbeitsvorgangs der Architekten bei der Siedlungsplanung aufnehmen und verbessern sollte. Das ist für einen solchen Zeitraum nicht möglich. Statt die vielen Ideen zu finden wäre es besser gewesen, sich am Anfang auf die Kernkompetenzen von MArC zu konzentrieren und bei den totalen Grundlagen anzufangen. Anschließend wäre ein weiterer, schrittweiser Einbau von Features möglich, wenn die Zeit das hergegeben hätte.\\
Desweiteren waren die hochmotivierten Teammitglieder sehr schnell an der praktischen Arbeit. Es wäre besser, für jedes Teilproblem eine genaue Analyse und sehr genaue recherche zu betreiben, welche Methode benötigt wird, ob diese schoneinmal verwendet wurde und wenn ja wie diese umgesetzt wurde. Mit einem breiteren und tieferen Vorwissen wären einige Fehler leichter erkennbar gewesen. Zum Beispiel beharrte das Team auf dem markerlosem Tracking. Dass dies mit so wenig Merkmalen eines Aluminiummarkers nicht möglich war und dafür Muster auf dem Marker notwendig wären, damit diese immer eindeutig identifizierbar sind, war dem Team lange nicht in dieser Wichtigkeit klar. So wurde viel Zeit auf ein markerloses Tracking verwendet, welche besser an anderen Baustellen aufgehoben wäre. \\
Ähnlich diesem Problems fand das Team am Anfang das kostenlose Planungstool \textit{Bitrix24}, welches scheinbar schnell und problemlos das Planen solcher Projekte erlaubte. Dass Accounts und Projekte nach einiger Zeit der Nichtbenutzung einfach gelöscht werden wurde dabei völlig übersehen. Glücklicherweise gab es die Planungsunterlagen noch an anderer Stelle, dennoch hätte soetwas nicht passieren dürfen.\\
Bei der Planung wollte das Team drei Prototypen entwerfen. Die Meilensteine zum erreichen dieser Prototypen waren viel zu eng gesetzt, die möglichen technischen Schwierigkeiten dabei unterschätzt. Hierbei hätte man auf die Erfahrung anderer Kommilitonen bzw. Professoren setzen müssen, um eine realistischere Einteilung zu schaffen.\\
Es wurde auf Hardware gesetzt, die den Ansprüchen des Projektes nicht gerecht wurde. So hatte die \textit{OVR Vision} im Projekt ein Hardwaredefekt und musste ausgetauscht werden. Das Team hätte sich früher gegen die Lösung mit dieser Kamera entscheiden müssen.\\
Die uEye hat eine zu geringe Framerate um ein absolut flüssiges Tracking zu ermöglichen. Dies hätte auch im Vorhinein durch gründliche Recherche auffallen müssen und eine alternative Kamera verwendet werden müssen.\\
Die Wichtigkeit der Kalibrierung der Kameras war dem Team lange nicht bewusst. Auch in diesem Thema hätte das Team frühzeitig tiefergehend recherchieren müssen und sich informieren müssen. Dies war bis zum Projektende ein Problem, welches das Team nur schwer entgegen kommen konnte und am Ende zeitliche Probleme bekam.\\
Auch auf programmiertechnischer Ebene gab es einige Herausforderungen. So wurden nicht von Anfang an sämtliche Klassen die gebraucht werden identifiziert und als Interface vorgearbeitet, sondern implementiert wie gerade notwendig war. \\
Alles in allem kann das Team aus dieser Zeit also viele Erfahrungen mitnehmen. Ist auch nicht alles glatt gelaufen, so ist \textit{MArC} dennoch entstanden und kann am 10.5.2017 präsentiert werden.


























