\section{Einleitung}
Virtuelle und erweiterte Realität ist bereits seit einiger Zeit in aller Munde. Mit dem Erscheinen von Oculus Rift, HTC Vive und anderen Virtual-Reality-Headsets rückt eine neue Art der Immersion beim Genuss von Videospielen in greifbare Nähe.\\
Wenn es jedoch um die Nutzung dieser Technologien zur Effizienzsteigerung in professionellen Umgebungen geht, so sind verfügbare Anwendungen bisher nur selten anzutreffen.
Die Entwicklung von MArC, einem Mixed-Reality-System für die architektonische Planung bei Siedlungsbauten, soll dies ändern.
\subsection{Anwendungskontext}
Die frühe Konzeptionierung zur Erschließung von Wohngebieten findet heute in Architekturbüros meist noch so statt wie vor dem Einzug der weit verbreiteten, digitalen Technik in unsere Leben. Dazu werden simple Modelle aus leicht zu verarbeitenden Materialien -- wie etwa Styropor -- erstellt und als Platzhalter für die zu planenden Gebäude bei dem Entwurf verwendet.\\
Diese Herangehensweise macht Änderungen an den Gebäuden aufwendig und führt zu dem Umstand, dass ein bestimmter Zustand der Planung nur umständlich wiederhergestellt werden kann -- zum Beispiel durch Fotografieren und späteren manuellen Wiederaufbau.
\subsection{Projektziel}
An diesem Punkt setzt MArC an, der "`Mixed Reality Architecture Composer"'. 
\todo[inline]{Vervollständigen}