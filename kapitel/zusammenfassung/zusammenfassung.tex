\section{Zusammenfassung}\todo[inline, color=green]{Lukas}
\todo[inline, color=red]{Paul}
In der vorliegenden Dokumentation wurde \emph{MArC}, der "`Mixed Reality Architecture Composer"' vorgestellt und detailliert beschrieben.\\
Zunächst erfolgte dazu in Kapitel~\ref{sec:Einleitung} die Einordnung des Projektvorhabens in die Ziel-Umgebung, außerdem wurde das Ziel des Projekts erläutert. Das Produkt \emph{MArC} sollte ein Werkzeug sein, welches Architekten die frühe Entwicklung von Siedlungen erleichtert. Auf architektonische Details der einzelnen Gebäude wurde dabei bewusst verzichtet, da \emph{MArC} vor allem bei der Platzierung der Gebäude behilflich sein soll.\\
In Kapitel~\ref{sec:grundlagen} wurden Grundlagen erörtert, die für die nachfolgenden Teile der Dokumentation für den Leser relevant sind. Diese Grundlagen beinhalten Theorie für die Entwicklung von Virtual-Reality-Anwendungen und damit verbundene Interaktionsmethoden, den grundlegenden Aufbau von Netz\-werk-Kom\-mu\-ni\-ka\-tions\-systemen, sowie eine Einführung in markerbasierte Objektverfolgung.\\
Alle während des Projekts verwendeten und erstellten Ressourcen, insbesondere Hard- und Software, wurden im Detail in Kapitel~\ref{sec:Materialien} thematisiert. Auch die in der ursprünglichen Projektplanung vorgesehene, später aber doch nicht verwendete Hardware wird dort beschrieben.\\
In Kapitel~\ref{sec:system} wurde das während der Projektlaufzeit entwickelte System vorgestellt. Neben dem grundsätzlichen Aufbau und der Verwendung des Systems durch einen Benutzer wurden auch die Netz\-werk-Kom\-mu\-ni\-ka\-tion der beiden Haupt-Soft\-ware\-kom\-po\-nen\-ten, die Menüführung vor dem Starten der \emph{Unity}-Simulation, sowie die Interaktion mit dem System als VR-Anwendung beschrieben.\\
Dem Tracking Programm als einem der Hauptteile von \emph{MArC} wurde das Kapitel~\ref{sec:Tracking} gewidmet. Darin wird sowohl die Anbindung der Kamera über dem Arbeitsbereich an die Tracking-Anwendung beschrieben, als auch die Kalibrierung des Systems mit den Koordinatentransformationen thematisiert, die nötig sind, um die Positionen der Würfel-Marker in den Koordinatenraum der \emph{Unity}-Simulation zu überführen. Des weiteren wird der Umgang der Tracking-Anwendung mit der Detektion, Verfolgung und Registrierung der verwendeten Marker behandelt.\\
In Kapitel~\ref{sec:ausblick} wurde ein kurzer Ausblick darüber gegeben, welche dem Projekt folgenden, weiteren Entwicklungsmaßnahmen von \emph{MArC} das Projektteam als sinnvoll erachtet. Dies beinhaltet neben Verbesserungen des verwendeten Marker-Tracking-Systems eine Erweiterung von \emph{MArC} mit einer auf dem Head-Mounted-Display montierten Kamera zu einer Augmented-Reality-Anwendung, wie sie zu Beginn dieses Projekts geplant war. Im Ausblick wird zusätzlich der Im- und Export von 3D-Modell-Daten für die Interoperabilität von \emph{MArC} mit Modellierungs- und Entwicklungssoftware behandelt, welche in Architekturbüros Anwendung findet. Außerdem wird eine Studie zur Gebrauchstauglichkeit von \emph{MArC} empfohlen, um insbesondere die Interaktionsmethoden der VR-Anwendung daraufhin zu überprüfen, ob diese tatsächlich zu einer gesteigerten Effizienz für Architekten bei der Siedlungsentwicklung führen.\\
Mit dem Projektmanagement, welches bei einem Masterprojekt wie \emph{MArC} eine zentrale Rolle spielt, befasst sich das Kapitel~\ref{sec:pm}. Dort wird die zu Beginn des Projekts erarbeitete Projektdefinition ebenso thematisiert wie die Projektplanung mit Arbeitspaketen, Meilensteinen und verschiedenen Plänen, die helfen sollen, die Planung und den Fortschritt des Projekts zu quantifizieren. Der Durchführung des Projekts mit dem Fokus auf der Zusammenarbeit im Projektteam und dem Projektabschluss sind ebenfalls Abschnitte gewidmet. Als weiteren wichtigen Teil der Analyse des Projekts ist die Reflexion in Abschnitt~\ref{sec:reflexion} anzusehen. Da die Erwartungen an das Projekt \emph{MArC} mehrfach während der Projektdauer angepasst und gesenkt werden mussten, weil viele unerwartete~--~und teils selbst verschuldete~--~Probleme auftraten, war es im Interesse des Projektteams, das Projekt besonders im Hinblick auf diese Schwierigkeiten kritisch zu analysieren und diese Reflexion in die Dokumentation mit einfließen zu lassen.

\section{Steckbrief}\todo[inline]{Vera} \todo[inline, color=blue]{Paul}