\section{Zusammenfassung}\todo[inline, color=yellow]{Lukas}
In der vorliegenden Dokumentation wurde \emph{MArC}, der "`Mixed Reality Architecture Composer"' vorgestellt und detailliert beschrieben.\\
Zunächst erfolgte dazu in Kapitel~\ref{sec:Einleitung} die Einordnung des Projektvorhabens in die Ziel-Umgebung, außerdem wurde das Ziel des Projekts erläutert.\\
In Kapitel~\ref{sec:grundlagen} wurden Grundlagen erörtert, die für die nachfolgenden Teile der Dokumentation notwendig sind. Diese Grundlagen beinhalten Theorie für die Entwicklung von Virtual-Reality-Anwendungen und damit verbundene Interaktionsmethoden, den grundlegenden Aufbau von Netz\-werk-Kom\-mu\-ni\-ka\-tions\-systemen, sowie eine Einführung in markerbasierte Objektverfolgung.\\
Alle während des Projekts verwendeten und erstellten Ressourcen, insbesondere Hard- und Software, wurden im Detail in Kapitel~\ref{sec:Materialien} thematisiert. Auch die in der ursprünglichen Projektplanung vorgesehene, später aber doch nicht verwendete Hardware wird dort kurz beschrieben.\\
In Kapitel~\ref{sec:system} wurde das während der Projektlaufzeit entwickelte System vorgestellt. Neben dem grundsätzlichen Aufbau und der Verwendung des Systems durch einen Benutzer wurden auch die Netzwerk-Kommunikation der beiden Haupt-Soft\-ware\-kom\-po\-nen\-ten, die Menüführung zum Starten der \emph{Unity}-Simulation, sowie die Interaktion mit dem System als VR-Anwendung beschrieben.\\
Dem Tracking Programm als einem der Hauptteile von \emph{MArC} wurde das Kapitel~\ref{sec:Tracking} gewidmet. Darin wird sowohl die Anbindung der Kamera über dem Arbeitsbereich an die Tracking-Anwendung beschrieben, als auch die Kalibrierung des Systems mit den notwendigen Koordinatentransformationen thematisiert, die nötig sind, um die Positionen der Würfel-Marker in den Koordinatenraum der \emph{Unity}-Simulation zu überführen. Des weiteren wird der Umgang der Tracking-Anwendung mit der Detektion, Verfolgung und Registrierung der verwendeten Marker behandelt.\\

\section{Steckbrief}\todo[inline]{Vera}