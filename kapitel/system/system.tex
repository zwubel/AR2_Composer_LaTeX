\section{System}
Die Benutzung von MArC ist in der dem Programm mitgelieferten ReadMe-Datei beschrieben. Darin wird erklärt, welche Hard- und Software komponenten erforderlich sind, wie das System gestartet und kalibriert wird. Des weiteren enthält die ReadMe eine Übersicht über die enthaltenen Quellcode-Dateien.
\subsection{Starten des Systems}
Nachdem sichergestellt wurde, dass alle in der ReadMe-Datei beschriebenen Voraussetzungen bestehen, kann das System gestartet werden, indem zunächst die Tracking-Anwendung (auf dem einen Computer) und anschließend die aus Unity heraus erstellte Anwendung (auf dem anderen Computer) gestartet wird. Auf letzterem Computer beginnt darauffolgend die Menüführung, welche in \ref{sec:menu} beschrieben ist. 
\subsection{Menüführung}\label{sec:menu}
Die Menüführung dient dazu, den Benutzer durch alle notwendigen Schritte zu leiten, die vor dem Starten der eigentlichen Simulation erforderlich sind. Im nachfolgenden Abschnitt~\ref{sec:menus} werden alle verfügbaren Menüs der Anwendung aufgelistet und kurz beschrieben, während im Abschnitt~\ref{sec:menuAblauf} der Ablauf der Menüführung erläutert wird.

\subsubsection{Menüs}\label{sec:menus}
Die folgenden Menüs sind Bestandteil der Menüführung:\todo[inline]{Screenshots der Menüs einfügen?}
\begin{description}
	\item[\texttt{CalibDone}:] Wird aufgerufen, wenn die Kalibrierung des Arbeitsbereichs abgeschlossen ist. Es informiert den Benutzer, dass die Kalibrierung erfolgreich war und der Vorgang fortgesetzt werden kann.
	\item[\texttt{CalibrateOrNot}:] Erscheint nach dem Verlassen des \texttt{Welcome}-Menüs und erlaubt dem Benutzer eine Kalibrierung durchzuführen oder eine bereits durchgeführte Kalibrierung zu laden.
	\item[\texttt{ControllerNotFound}:] Warnt den Benutzer nach dem Starten der Kalibrierung, dass der HTC Vive Controller, welcher für die Kalibrierung benötigt wird, nicht eingeschaltet ist. Während das Menü angezeigt wird, kann der Benutzer den Controller einschalten und anschließend auf \texttt{Continue} klicken.
	\item[\texttt{doPlaneCalibInVS}:] Dient dem Benutzer als Anleitung für die Durchführung der Arbeitsbereich-Kalibrierung. Diese wird in \ref{sec:planeCalib} genauer beschrieben.
	\item[\texttt{doPoseCalibInVS}:] Dient dem Benutzer als Anleitung für die Durchführung der Kamera-Kalibrierung. Diese wird in \ref{sec:camCalib} genauer beschrieben.
	\item[\texttt{SelectCalibrationTarget}:] Erlaubt die Auswahl der Art der Kalibrierung. Es kann hier entweder nur der Arbeitsbereich oder sowohl der Arbeitsbereich, als auch die Kamera kalibriert werden. Die Kalibrierung ist näher in \ref{sec:calib} beschrieben.
	\item[\texttt{SetScale}:] Stellt das letzte Menü vor dem Starten der Simulation dar. In diesem kann der Benutzer den Maßstab der Gebäudesimulation einstellen und anschließend die Simulation starten.
	\item[\texttt{SocketNotReady}:] Warnt den Benutzer nach dem Verlassen des \texttt{Welcome}-Menüs, dass die Netzwerkverbindung zum Computer, auf dem die Tracking-Anwendung ausgeführt wird, nicht bereit ist. Nach Bestätigung dieses Hinweises durch einen Klick auf \texttt{Continue}, kehrt der Benutzer zum \texttt{Welcome}-Menü zurück. Anschließend kann der Vorgang fortgesetzt werden, wenn die Netzwerkverbindung hergestellt wurde. Andernfalls erscheint wieder \texttt{SocketNotReady}.
	\item[\texttt{Welcome}:] Erscheint als erstes Menü. Hier erhält der Nutzer eine kurze Information darüber, wie die Anwendung heißt und wozu sie dient.
\end{description}

\subsubsection{Ablauf der Menüführung}\label{sec:menuAblauf}
Der Ablauf der Menüführung von MArC ist in Abbildung~\ref{fig:menuFlow} dargestellt. Die einzelnen Menüs sind bereits in~\ref{sec:menus} beschrieben worden.

Nach dem Starten der Anwendung wird zunächst das Menü \texttt{Welcome} angezeigt. Dieses enthält nur einen Button \textit{Get started}. Sobald dieser gedrückt wird, prüft die Anwendung, ob eine Netzwerkverbindung zu dem Computer mit der Tracking-Anwendung besteht. Sollte dies nicht der Fall sein, wird das Menü \texttt{Socket\-Not\-Ready} angezeigt. Dieses verlässt der Benutzer über einen Klick auf \texttt{Continue}, anschließend wird erneut das Menu \texttt{Welcome} angezeigt. Wenn zu diesem Zeitpunkt die Netzwerkverbindung korrekt hergestellt wurde, gelangt der Benutzer zum Menü \texttt{Calibrate\-Or\-Not}, anderenfalls wird wiederholt \texttt{Socket\-Not\-Ready} angezeigt.

In \texttt{CalibrateOrNot} hat der Benutzer die Auswahl zwischen den Schaltflächen \textit{Yes} und \textit{No}. Bei einem Klick auf \textit{Yes} wird anschließend \texttt{Select\-Calibration\-Target} angezeigt, bei einem Klick auf \textit{No} lädt das System eine zuvor durchgeführte Kalibrierung und das Menü \texttt{Set\-Scale} wird geöffnet.

\texttt{Select\-Calibration\-Target} stellt den Benutzer vor die Wahl entweder nur den Arbeitsbereich (\textit{Work\-space}) oder sowohl den Arbeitsbereich als auch die Kamera zu kalibrieren (\textit{Camera and Workspace}). Außerdem besteht die Möglichkeit über \textit{Cancel} zum Menü \texttt{Calibrate\-Or\-Not} zurückzukehren.

Wählt der Benutzer \textit{Camera and Workspace} in \texttt{Select\-Calibration\-Target} aus, so informiert die Anwendung die Tracking-Anwendung auf dem anderen Computer und wartet anschließend darauf, dass von dort die Bestätigung gesendet wird, dass die Kamerakalibrierung abgeschlossen ist. Anschließend wird das Menü \texttt{doPlane\-Calibration\-InVS} angezeigt, welches auch aufgerufen wird, wenn der Benutzer \textit{Work\-space} in \texttt{Select\-Calibration\-Target} wählt.

Im Menü \texttt{doPlane\-Calibration\-InVS} wird zunächst geprüft, ob der für die Kalibrierung notwendige HTC Vive Controller eingeschaltet ist. Sollte dies nicht der Fall sein, wird \texttt{Controller\-Not\-Found} aufgerufen. Dieses kann mit einem Klick auf \textit{Continue} verlassen werden, woraufhin wieder \texttt{Select\-Calibration\-Target} angezeigt wird.\\
Sofern der HTC Vive Controller beim Aufruf von \texttt{doPlane\-Calibration\-InVS} eingeschaltet ist, wird nach Durchführung der Kalibrierung des Arbeitsbereichs das Menü \texttt{CalibDone} angezeigt.

\texttt{Calib\-Done} kann über einen Klick auf \textit{Continue} verlassen werden und führt den Nutzer anschließend zu \texttt{SetScale}. Aus diesem Menü kann über den Button \textit{Back} entweder zu \texttt{CalibrateOrNot} zurückgekehrt oder die Simulation mit dem im Menü über den Slider eingestellten Maßstab gestartet werden.

\begin{figure}[htbp]
	\centering
	\includegraphics[scale=.9, trim=5.5cm 2.5cm 3.5cm 2.5cm]{kapitel/system/MP_Menu_Flowchart.pdf}
	\caption{Flussdiagramm der Menüführung.}
	\label{fig:menuFlow}
\end{figure}
\subsection{Kalibrierung}\label{sec:calib}
Um das Ziel von MArC zu erreichen, an den Positionen der Aluminiumwürfel im Arbeitsbereich in der virtuellen Realität von Unity gerenderte Würfel darzustellen, muss das System kalibriert werden. Die Kalibrierung hat zum Ziel, eine Koordinatentransformation zu finden, die Positionen im Kamera-Koordinatensystem in das Unity-Koordinatensystem transformiert.

Zu diesem Zweck muss eine zweistufige Kalibrierung durchgeführt werden. Zunächst sorgt die Kamerakalibrierung dafür, dass Bildkoordinaten auf dem Sensor der Kamera in das 3D-Kamera-Koordinatensystem transformiert werden. Dafür wird sich einiger OpenCV-Funktionen in Verbindung mit AruCo-Markern bedient. Dieser Vorgang wird nachfolgend in~\ref{sec:camCalib} genauer beschrieben.\\
Der nächste Schritt, die Kalibrierung des Arbeitsbereichs, bestimmt über Punkt-Korrespondenzen -- also in zwei verschiedenen Koordinatensystemen bekannte Punkte -- eine affine 3D-Transformation, welche die Abbildung vom Kamera-Koordinaten\-system auf das Unity-Koordinatensystem ermöglicht. Dieser Kalibrierungsschritt wird nachfolgend in~\ref{sec:planeCalib} näher beschrieben.
\subsubsection{Kamerakalibrierung}\label{sec:camCalib}
\subsubsection{Kalibrierung des Arbeitsbereichs}\label{sec:planeCalib}


