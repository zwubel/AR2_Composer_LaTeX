\section{Materialien}
\subsection{Hardware}
\subsubsection{Rechner 1}
Noch nicht klar.
\subsubsection{Rechner 2 ??}
Noch nicht klar.
\subsubsection{HTC Vive}
Bei der HTC Vive handelt es sich um ein Head-Mounted Display, welches von HTC in Kooperation mit Valve \cite{website:Valve} produziert wird. Vorgestellt wurde diese am 1. März 2015 im Vorfeld der Mobile World Congress. Die offizielle Markteinführung fand dann im April 2016 statt, nachdem eine neue Generation der VR Brille auf der CES 2016 vorgestellt wurde. \\
Die Auflösung des Displays beträgt insgesamt 2160x1200 Pixel, was 1080x1200 pro Auge enstpricht. Die Brille bietet ein Sichtfeld von bis 110$^\circ$ bei einer Bildwiederholrate von 90 Hz \cite{website:HTC_Vive}. Zur Positionsbestimmung im Raum wird die  Lighthousetechnologie von Valve genutzt. Zusätzlich sind neben einem Gyrosensor auch ein Beschleunigungsmesser   und ein Laser-Positionsmesser verbaut. Mittels speziellen Game-Controllern wird eine Interaktion mit virtuellen Objekten ermöglicht. Die eingebaute Frontkamera wird für dieses Projekt nicht verwendet. Stattdessen wird auf die OVR Vision zurückgegriffen, die im Folgenden beschrieben wird.
\subsubsection{OVR Vision}
\subsubsection{Ueye}
\subsubsection{Leap Motion ??}
\subsubsection{Marker}
\subsection{Software}
\subsubsection{Unity}
\subsubsection{Visual Studio}
\subsubsection{OpenCV}


\newpage